Zaimplementowany~w~aplikacji sensor LVDT należy do grupy czujników wykorzystujących zmiany
parametrów obwodu -- indukcyjności. Czujnik typu LVDT (ang. Linear Variable Differential
Transformer, transformatorowy czujnik przemieszczeń liniowych) zbudowany jest na podstawie
transformatora różnicowego. Posiada on trzy uzwojenia -- pierwotne oraz dwa wtórne, które nawijane
są na cylindryczną, nieprzewodzącą obudowę. Wewnątrz obudowy znajduje się rdzeń ferromagnetyczny,
który może się swobodnie poruszać. Schemat układu przetwornika LVDT przedstawiono na
rysunku~\ref{img:lvdt}.

\begin{figure}[!htbp]
  \centering
  \includegraphics[width=0.4\textwidth]{sensor-theory/displacement-lvdt}
  \caption{\label{img:lvdt}Schemat układu przetwornika LVDT}
\end{figure}

Uzwojenie pierwotne zasilane jest napięciem przemiennym.~W~uzwojeniach wtórych (oznaczone jako Ns na
schemacie) indukują się napięcia, które zależne są od położenia rdzenia wewnątrz obudowy.~W~stanie
ustalonym -- gdy rdzeń znajduje się~w~równej odległości od obu cewek wtórnych -- napięcia będą sobie
równe. Ponieważ układ sensora LVDT jest zbudowany na podstawie transformatora różnicowego, to
napięcie na wyjściu całego układu będzie równe różnicy napięć na cewkach wtórnych. Zależność tę
opisuje wzór \cite{sensory_wykład}:

\begin{equation}\label{eqn:theory-lvdt}
  U_{wy}=f\cdot I_p\cdot\bigg(4\pi\cdot N_p N_w\cdot \mu_0\cdot l_p\cdot\frac{x}{3l_w}\cdot
  \log{\Big(\frac{r_o}{r_i}\Big)}\bigg)\bigg(1-\frac{x^2}{2l_p^2}\bigg)
\end{equation}

\begin{eqparams}
  f & częstotliwość zasilania układu,\\
  I_p & prąd~w~uzwojeniu pierwotnym $I_p=\cfrac{U_{wej}}{R}$,\\
  N_p,\,N_w & liczba zwojów~w~uzwojeniach, odpowiednio~w~pierwotnym~i~wtórnych,\\
  l_p,\,l_w & długości uzwojeń, odpowiednio pierwotnego~i~wtórnych,\\
  \mu_0 & przenikalność magnetyczna próżni ($4\pi\cdot 10^{-7}$),\\
  x & przesunięcie rdzenia względem stanu ustalonego,\\
  \cfrac{r_o}{r_i} & stosunek promieni zewnętrzynych~i~wewnętrznych układu cewki,\\
\end{eqparams}

Charakterystyka przetwarzania sensora, przedstawiona na rysunku \ref{img:transfer-lvdt}, ma kształt
litery ,,V'', obie strony są symetryczne względem punktu zerowego -- stanu zrównoważenia. Dodatkowo,
napięcie wyjściowe nie jest zerowe~w~stanie równowagi -- jest to spowodowane niewielkimi różnicami
pomiędzy uzwojeniami wtórnymi \cite{sensory_wykład}.

\begin{figure}[!htbp]
  \centering
  \includegraphics[width=0.9\textwidth]{sensor-theory/transfer-lvdt}
  \caption{\label{img:transfer-lvdt}Charakterystyka przetwarzania sensora LVDT~w~zależności od
    przesunięcia.}
\end{figure}