Sensory termoelektryczne wykorzystują zjawisko termoelekryczne, nazywane też efektem
Seebecka.~W~obwodzie, który składa się~z~dwóch różnych metali, których końce są połączone~i~znajdują
się~w~różnych temperaturach, płynie prąd elektryczny. Jednakże~w~celu pomiarów temperatury wymagany
jest pomiar różnicy potencjałów (siły elektrotermicznej), pojawiającej się między niepołączonymi
końcami przewodników. Działanie zjawiska przedstawione zostało na rysunku
\ref{img:thermoelectric-effect}. Zmierzona różnica potencjałów wyraża się równaniem:

\begin{equation}\label{eqn:voltage-thermocouple}
  E = S_{AB}\cdot\Delta T
\end{equation}

\begin{eqparams}
  S_{AB} & współczynnik Seebecka (~w~temperaturze odniesienia 0\degC), \\
  \Delta T & różnica temperatur pomiędzy stroną \enquote{gorącą} oraz \enquote{zimną} układu.
\end{eqparams}

\addimage{0.6}{sensor-theory/thermoelectric-effect}{\label{img:thermoelectric-effect}Schemat
  działania zjawiska termoeletrycznego}

~W~praktyce przy pomiarach termoogniwami wykorzystuje się układ przedstawiony na
rysunku~\ref{img:thermocouple}. Jeżeli do połączenie wykorzystane zostałyby przewody
wykonane~z~miedzi, to~w~miejscu połączenia pojawiłby się kolejne złącza termoelekryczne. Nie
pojawiłby się tutaj żaden błąd pomiarowy, ponieważ przy takim połączniu występuje tzw. ,,prawo
trzeciego metalu'' -- dodatkowe siły termoelektryczne redukują się. Mimo to~w~układach
pomiarowych~z~termoogniwami stosuje się specjalne przewody kompensacyjne, które wykonane
są~z~takiego samego materiału jak przetwornik lub~z~materiałów~o~takich samych własnościach
termoelektrycznych. W celu zapewnienia dokładności pomiarowej ważne jest zapewnienie tej samej
temperatury odniesienia na obu końcach ,,zimnej'' strony układu pomiarowego. Dlatego punkt ten
umieszcza się w środowisku izolowanym termicznie, gdzie znana jest wartość temperatury.

\addimage{0.6}{sensor-theory/thermocouple}{\label{img:thermocouple}Schemat
  układu do pomiarów termoogniwami}

Współczynnik Seebecka jest wartością charakterystyczną dla danego typu termoogniwa, zależnym od
metali, które zostały użyte do wytworzenia sensora. Wartości współczynnika \cite{thermocouple} dla
wybranych termoelementów wraz~z~ich oznaczeniami przedstawia tabela \ref{tab:thermocouple}.
Pomiary~z~wykorzystaniem termoogniw można wykonywać~w~zakresie -250 do 1760\degC,~w~zależności od
użytego typu termoogniwa. Zakresy temperatur dla każdego~z~typów także przedstawione
zostały~w~tabeli. W~aplikacji zaimplementowany został zakres 0 do 500\degC, tak aby łatwiej można
było zaobserwować różnice pomiędzy termoogniwami~a~termorezystorami.

\begin{table}[!htbp]
  \centering
  \caption{\label{tab:thermocouple}Wartości wsp. Seebecka oraz zakresy temperatur dla wybranych
    termoogniw}
  \begin{tabular}{cccc}
    \toprule
    Typ termoogniwa & Oznaczenie & Wsp.Seebecka $S_{AB}$ [$\mu V\cdot$\degC$^{-1}$] & Zakres temperatur [\degC] \\
    \midrule
    Fe-CuNi         & J          & 51                                               & 0 $\div$ 750              \\
    NiCr-NiAl       & K          & 40                                               & -200 $\div$ 1250          \\
    PtRh(13\%)-Pt   & R          & 12                                               & 0 $\div$ 1450             \\
    Cu-CuNi         & T          & 60                                               & -250 $\div$ 350           \\
    NiCr-CuNi       & E          & 40                                               & -200 $\div$ 900           \\
    \bottomrule
  \end{tabular}
\end{table}

Wartości współczynnika Seebecka podane~w~tabeli \ref{tab:thermocouple} są wartościami uśrednionymi
dla temperatury odniesienia 273.15K (0\degC). Aplikacja wykorzystuje takie wartości, ponieważ tylko
symuluje realne przetworniki. Przy normalnych pomiarach nie możnaby przyjąć stałej wartości, gdyż
jest ona zależna od temperatury otoczenia (wartości współczynnika nie będą stałe~w~całym zakresie
pomiarowym danego termoogniwa).