Na zmianę parametrów czujnika tensometrycznego metalowego największy wpływ ma wartość odkszałcenia
elementu, na którym zamontowany jest sensor. Jednakże istotną rolę dla pracy przetwornika stanowi
wpływ temperatury otoczenia. Negatywny wpływ temperatury można skompensować poprzez odpowiedni dobór
parametrów tensometru na etapie produkcji -- dodając do podstawowego materiału sensora odpowiednie
domieszki. Przetworniki tensometryczne przetwarzają nieelektryczną wielkość odkształcenia
$\varepsilon$ na wielkość elektryczną -- zmianę rezystancji. Następnie~w~torze pomiarowym stosuje
się przetwornik pozwalający na generowanie mierzalnego sygnału napięciowego.

Jeżeli element, na którym zamontowane są tensometry ulegnie odkształceniu, to zmieni się wartość
rezystancji sensora. Wartości zmian rezystancji mogą być dodatnie lub ujemne, jest to zależne od
kierunku~w~jakim odkształca się czujnik. Przedstawia to wzór \cite{gawedzki2010}:

\begin{equation}
  R_i = R_{i\,0} \pm \Delta R_i, \quad\quad\quad \text{gdzie:}\quad i=1,2,3,4
\end{equation}

Zmianę wartości rezystancji przetwornika tensometrycznego przedstawia zależność opisana wzorem
\ref{eqn:strain-r}, natomiast~w~przypadku, gdy brany jest pod uwagę wpływ temperatury zależność ta
przedstawiona jest wzorem \ref{eqn:strain-temp}.

\begin{equation}\label{eqn:strain-r}
  \Delta R_i = R_i\cdot\varepsilon\cdot k
\end{equation}

\begin{equation}\label{eqn:strain-temp}
  \Delta R_i = R_i(\varepsilon\cdot k + \alpha\Delta T)
\end{equation}

\begin{eqparams}
  R_i & bazowa rezystancja tensometru,\\
  \varepsilon & wartość odkształcenia,\\
  k & współczynnik odkształcenia (stała tensometryczna)\\
  \alpha & współczynnik temperaturowy metalu (zgodnie~z~tabelą \ref{tab:temp_alpha}),\\
  \Delta T & różnica temperatur pomiędzy temperaturą otoczenia~a~odniesienia (20\degC),\\
\end{eqparams}

Do pomiarów tensometrycznych często stosuje się układy mostkowe. Zastowanie takiego układu pozwala
eliminację składowej siły powodującej odkształcenie, która nie jest mierzona. Dodatkowo dzięki
zastosowaniu mostka możliwa jest kompensacja zakłóceń związanych ze zmianami temperatury otoczenia
podczas pomiaru.

Wyróżnia się trzy konfiguracje mostka tensometrycznego:
\begin{itemize}
  \item [--] układ ćwierćmostkowy~z~jednym pracującym tensometrem
  \item [--] układ półmostkowy~z~dwoma czynnymi tensometrami (występują dwa rodzaje tej
        konfiguracji,~z~tensometrami~w~ramionach sąsiednich lub przeciwległych),
  \item [--] układ pełnomostkowy~z~czterema aktywnymi tensometrami.
\end{itemize}
Wszystkie układy mają różną skuteczność kompensacji wpływu temperatury. Na
rysunku~\ref{img:strain-gauge} przedstawiony został schemat układu mostkowego.

\addimage{0.65}{sensor-theory/strain-gauge}{\label{img:strain-gauge}Schemat układu mostkowego}

\paragraph{Układ ćwierćmostkowy:} Posiada jeden czynny tensometr (przykładowo~w~pierwszym ramieniu
mostka), do wyznaczenia napięcia na wyjściu układu, korzystając~z~zależności \ref{eqn:strain-temp},
stosuje się równanie:

\begin{equation}
  U_{wy}=\cfrac{1}{4}\cdot{U_z}\cdot\cfrac{\Delta R_1}{R_1}=\cfrac{1}{4}\cdot{U_z}\cdot({k}
  \cdot\varepsilon_1+{k_1}\cdot\alpha_1\Delta{T_1})
\end{equation}

Na podstawie tego równania można stwierdzić, że układ nie będzie kompensował wpływu temperatury.

\paragraph{Układ półmostkowy:} Posiada dwa czynne tensometry. Jeżeli zastostuje się
wariant~z~tensometrami~w~sąsiednich ramionach (przykładowo pierwszym~i~trzecim), to napięcie
wyjściowe wyznacza się~z~zależności:

\begin{equation}\label{eqn:half-bridge-adj}
  U_{wy}=\cfrac{1}{4}\cdot{U_z}\cdot\bigg(\cfrac{\Delta R_1}{R_1}-\cfrac{\Delta R_3}{R_3}\bigg)=
  \cfrac{1}{4}\cdot{U_z}\cdot\big({k}\cdot(\varepsilon_1-\varepsilon_3)+{k_1}\cdot(\alpha_1
  \Delta{T_1}-\alpha_3\Delta{T_3})\big)
\end{equation}

Jeżeli obecne~w~układzie tensometry będą pracować~w~takiej samej temperaturze, to wpływ temperatury
zostanie skompensowany~i~napięcie wyjściowe będzie można wyznaczyć korzystając~z~równania:

\begin{equation}\label{eqn:half-bridge-compensation}
  U_{wy}=\cfrac{1}{4}\cdot{U_z}\cdot\big({k}\cdot(\varepsilon_1-\varepsilon_3)\big)
\end{equation}

Natomiast~w~przypadku zastosowania wariantu~o~ramionach przeciwległych (przykładowo~z~tensometrami
włączonymi do ramion pierwszego~i~czwartego) to zależność \ref{eqn:half-bridge-adj} przyjmie postać:

\begin{equation}
  U_{wy}=\cfrac{1}{4}\cdot{U_z}\cdot\bigg(\cfrac{\Delta R_1}{R_1}-\cfrac{\Delta R_3}{R_3}\bigg)=
  \cfrac{1}{4}\cdot{U_z}\cdot\big({k}\cdot(\varepsilon_1+\varepsilon_3)+{k_1}\cdot(\alpha_1
  \Delta{T_1}+\alpha_3\Delta{T_3})\big)
\end{equation}
~i~nie nastąpi kompensacja, natomiast wynik pomiaru będzie zawierał błąd związany~z~negatywnym
wpływem temperatury.

\paragraph{Układ pełnomostkowy:} Posiada cztery tensometry włączone do każdego~z~ramion mostka.
Napięcie wyjściowe wyznacza się równaniem:

\begin{align}
  U_{wy} & =\cfrac{1}{4}\cdot{U_z}\cdot\bigg(\cfrac{\Delta R_1}{R_1}+\cfrac{\Delta R_2}{R_2}-
  \cfrac{\Delta R_3}{R_3}+\cfrac{\Delta R_4}{R_4}\bigg)=              \nonumber               \\
         & =\cfrac{1}{4}\cdot{U_z}\cdot\big({k}\cdot
  (\varepsilon_1+\varepsilon_2-\varepsilon_3+\varepsilon_4)+{k_1}\cdot(\alpha_1\Delta{T_1}+
  \alpha_2\Delta{T_2}-\alpha_3\Delta{T_3}+\alpha_4\Delta{T_4})\big)
\end{align}

Zakładając identyczne warunki termiczne~w~otoczeniu każdego sensora można stwierdzić, że wpływ
temperatury zostanie skompensowany.~W~takim przypadku napięcie wyjściowe będzie można wyznaczyć
korzystając~z~równania \ref{eqn:half-bridge-compensation} uzupełnionego~o~dwa dodatkowe
tensometry~w~układzie:

\begin{equation}
  U_{wy}=\cfrac{1}{4}\cdot{U_z}\cdot\big({k}\cdot(\varepsilon_1+\varepsilon_2-\varepsilon_3+
  \varepsilon_4)\big)
\end{equation}

\begin{eqparams}[W każdym~z~powyższych wzorów (\textit{\textrm{i}} = 1,2,3,4):]
  \Delta R_i & zmiana rezystancji i-tego tensometru (wartość może być dodatnia lub ujemna)\\
  R_i & bazowa rezystancja i-tego tensometru\\
  k & współczynnik odkształceń (stała tensometryczna)\\
  \varepsilon_i & odkształcenie i-tego tensometru\\
  U_z & napięcie zasilania układu mostkowego\\
  \alpha_i & współczynnik temperaturowy materiału i-tego tensometru\\
  \Delta{T_i} & różnica temperatur pomiędzy temperaturą otoczenia~a~odniesienia (20\degC)\\
\end{eqparams}

We wzorach pojawia się dodatkowa stała $k_1$, ponieważ rzeczywiste stałe tensometryczne
odbiegają od wartości założonych przez producenta sensora w fazie projektowania.

Problem braku kompensacji temperaturowej w układach ćwierćmostkowym oraz półmostkowym z ramionami
przeciwległymi można rozwiązać poprzez zastosowanie tensometrów biernych. Montuje się je w pobliżu
dwóch czynnych tensometrów, jednakże bez przytwierdzania ich do elementu, którego odkształcenia są
mierzone \cite{gawedzki2010}. W takiej konfiguracji sensory bierne nie zmieniają swojej rezystancji
na skutek działającej siły, a tylko na zmiany temperatury, dzięki czemu umożliwią kompensowanie jej
wpływu.


