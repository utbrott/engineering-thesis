W sensorach termorezystancyjnych metalowych, do których należy RTD, wykorzystywane jest zjawisko
zmiany rezystacji metalu~w~funkcji temperatury. Do najczęściej stosowanych należą metale: platyna
(Pt), nikiel (Ni), miedź (Cu), których charakterystyki~i~właściwości metrologiczne są
ujęte normie PN-EN 60751:2009 \cite{rtd_norm}.

Oznaczanie czujników temperatury jest związane~z~rodzajem zastosowanego metalu oraz wartością
rezystancji czujnika $R_0$~w~temperaturze odniesienia $T_0$, którą przyjmuję się jako 0\degC\space
\cite{gawedzki2010}. Zależność rezystancji od temperatury standardowo jest nieliniowa~i~wyznaczana
jest eksperymentalnie, jednakże można wykorzystać aproksymację funkcją liniową:

\begin{equation}\label{eqn:resistance-rtd}
  R_T=R_0(1+\alpha\cdot\Delta T)
\end{equation}

\begin{eqparams}
  R_0 & rezystancja czujnika~w~temperaturze odniesienia, \\
  \alpha & współczynnik temperaturowy metalu, \\
  \Delta T & różnica temperatur pomiędzy otoczeniem sensora~a~referencyjną.
\end{eqparams}

\noindent Współczynnik temperaturowy definuje się jako względna zmiana rezystancji spowodowana
zmianą temperatury, co przedstawia równanie \ref{eq:alpha1}, natomiast~w~przypadku liniowej
aproksymacji zależność przedstawia równanie \ref{eq:alpha2} \cite{gawedzki2010}.~W~tabeli
\ref{tab:temp_alpha} przedstawiono wartości współczynnika dla wybranych metali.

\begin{equation}\label{eq:alpha1}
  \alpha=\frac{dR_T}{R_T\cdot dT}
\end{equation}

\begin{equation}\label{eq:alpha2}
  \alpha=\frac{R_T-R_0}{R_0\cdot(T-T_0)}
\end{equation}

\begin{table}[!htbp]
  \centering
  \caption{\label{tab:temp_alpha}Wartości współczynnika temperaturowego dla wybranych metali}
  \begin{tabular}{cc}
    \toprule
    Metal            & Współczynnik temperaturowy $\alpha$ \\
    \midrule
    Platyna          & 0.003729                            \\
    Miedź            & 0.004041                            \\
    Nikiel           & 0.00617                             \\
    Wolfram          & 0.0045                              \\
    Konstanan (CuNi) & -0.000074                           \\
    Monel (NiCu)     & 0.0011                              \\
    \bottomrule
  \end{tabular}
\end{table}

Typowe układy pracy dla termorezystorów metalowych to:
\begin{itemize}
  \item [--] \textbf{układ 2-przewodowy}, który używa się~w~sytuacjach, gdy nie jest wymagana duża
        dokładność pomiarowa. Takie połączenie powoduje dodanie rezystancji przewodów do rezystancji
        czujnika, co skutkuje błędami pomiarowymi. Im większa długość przewodów~w~układzie, tym
        większe będą błędy pomiarowe;
  \item [--] \textbf{układ 3-przewodowy} wykorzystywany, aby zniwelować wpływ rezystacji przewodów
        na otrzymywane wyniki pomiarów.~W~takim połączeniu powstają dwa układy pomiarowe. Jeden z
        nich wykorzystywany jest do pomiaru wyłącznie rezystancji przewodów. Umożliwia to
        kompensację błędu pomiarowego;
  \item [--] \textbf{układ 4-przewodowy}, który jest najdokładniejszym~z~układów pomiarowych. W
        praktyce stosowany tylko~w~pomiarach laboratoryjnych. Otrzymywane wyniki pomiarowe
        pozbawione są jakiegokolwiek błędu związanego~z~przewodami~w~układzie.
\end{itemize}

Termorezystory metalowe stostuje się do pomiarów~w~zakresach -240 do 650\degC.~W~aplikacji
zaimplementowany został zakres 0 do 500\degC, tak aby łatwiej można było porównać je do termoogniw.