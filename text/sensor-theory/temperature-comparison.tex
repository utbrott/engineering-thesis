Oba typy sensorów posiadają charakterystyczne dla nich układy pomiarowe, zakresy w jakich można je
wykorzystać. Wiąże się to z szeregiem zalet oraz wad każdego z nich. Tabela
\ref{tab:rtd-vs-thermocouple} przedstawia porównanie wybranych parametrów obu wariantów sensorów
temperatury. natomiast w tabeli \ref{tab:thermocouple-pros-cons} przedstawione zostało porównanie
ich zalet i wad.
\begin{table}[!htbp]
  \centering
  \caption{\label{tab:rtd-vs-thermocouple}Porównanie parametrów termorezystorów do termoogniw}
  \begin{tabular}{ccc}
    \toprule
    Parametr                           & Termorezystor   & Termoogniwo        \\
    \midrule
    Liniowość charakterystyki          & Dobra           & Dostateczna        \\
    Stabilność                         & Dobra           & Dostateczna        \\
    Czułość sensora                    & Średnia         & Niska              \\
    Czas odpowiedzi                    & Średni          & Średni do niskiego \\
    Dokładność                         & Wysoka          & Średnia            \\
    Wytrzymałość mechaniczna           & Dobra           & Znakomita          \\
    Podatność na efekt samoogrzewania? & Tak, nieznaczna & Nie                \\
    Koszt sensora                      & Wysoki          & Niski              \\
    \bottomrule
  \end{tabular}
\end{table}

\begin{table}[!htbp]
  \centering
  \caption{\label{tab:thermocouple-pros-cons}Zalety i wady obu rodzajów przetworników}
  \begin{tabular}{p{0.5cm}|c|c}
    \toprule
                                                                       & Termorezystory                        & Termoogniwa                                \\
    \midrule
    \parbox[t]{2mm}{\multirow{3}{*}{\rotatebox[origin=c]{90}{Zalety}}} & Najbardziej stabilny typ sensora      & Prosta budowa, tanie w wykonaniu           \\
                                                                       & Liniowe charakterystyki przetwarzania & Nie wymagają zewnętrznego zasilania        \\
                                                                       & Najdokładniejsze pomiary              & Szeroki zakres temperatur                  \\
    \midrule
    \parbox[t]{2mm}{\multirow{4}{*}{\rotatebox[origin=c]{90}{Wady}}}   & Wymagają zewnętznego zasilania        & Nieliniowa charakterystyka przetwarzania   \\
                                                                       & Niewielkie zmiany rezystancji         & Niskie napięcie wyjściowe                  \\
                                                                       & Podatne na efekt samoogrzewania       & Mało stabilne pomiary                      \\
                                                                       &                                       & Wymagana znajomość temperatury odniesienia \\
    \bottomrule
  \end{tabular}
\end{table}

Termorezystory metalowe są podatne na efekt samoogrzewania. Jest on związany z wymogiem zewnętrznego
zasilania układu, którym pracuje sensor. Efekt ten, spowodowany przepływem prądu przez rezystor
jakim jest sensor, może powodować pojawianie się błędów pomiarowych \cite{self-heating}. Nie
występuje on w przypadku termoogniw, ponieważ ich układ nie jest zasilany zewnętrznie -- wykonuje
się w nim pomiary różnicy potencjałów między końcami przewodników użytych do zbudowania sensora.