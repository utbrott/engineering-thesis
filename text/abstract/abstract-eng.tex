The aim of this thesis was to design and implement web application, that can be used as~a~tool to
support students performing excercises in the Measurment Transducers laboratory. The application was
created using TypeScript programming language, which is an extension, a superset, to JavaScript
programming language.

The tool has implemented 3 types of measurement sensors, which are used in measurements of
non-electrical quantities -- temperature (resistance temperature detectors, thermocouples),
displacement (linear variable differential transformers) and stress (strain gauge) sensors. It is
possible to configure the sensor that was selected for testing, based on the real measuring
transducers parameters. Application gives access to tasks, that the user needs to complete to
understand how certain sensor works. In addition to that it is possible to view charts with
characteristics generated by the app, that can be downloaded and saved for later use.

The web application was built using the \textit{React} library and the \textit{Next.js} framework,
which give the ability to create complex and interactive applications. Using TypeScript programming
language to write applications' source code gave the ability to use the tools that are not available
when using standard web application programming language -- JavaScript. The user interface was
created with the help of \textit{Chakra UI} library. It gives access to ready-made components, which
after additional customization allow skipping the manual styling of each component in the
application. Smaller libraries were also used, which helped with adding mathematical formulas and
generating graphs and charts. Every tool and library used belonged to the Open Source software
family -- they are free to use, one can edit their source code to own needs in accordance to the
attached licence.