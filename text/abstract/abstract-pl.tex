Celem pracy było stworzenie aplikacji internetowej, która może zostać wykorzystana jako narzędzie do
wspomagania studentów wykonujących ćwiczenia~w~laboratorium Przetworników Pomiarowych. Aplikacja
napisana jest~w~języku TypeScript, który jest rozszerzeniem języka JavaScript.

Narzędzie posiada zaimplementowane 3 typy sensorów pomiarowych, które wykorzystuje się~w~pomiarach
wielkości nieelektrycznych -- sensory temperatury (termorezystory metalowe, termoogniwa),
przemieszczenia (transformatorowe czujniki przemieszczeń liniowych) oraz naprężeń
(tensometry). Dostępna jest możliwość skonfigurowania wybranego do badań
sensora~w~oparciu~o~parametry rzeczywistych przetworników pomiarowych. Aplikacja daje także dostęp
do zadań, które należy wykonać~w~celu zrozumienia jak działa dany czujnik. Dodatkowo użytkownik ma
także dostęp do wygenerowanych charakterystyk, które mogą zostać pobrane~i~zapisane~w~celu
późniejszego wykorzystania.

Aplikacja zbudowana została wykorzystując bibliotekę \textit{React} oraz framework \textit{Next.js},
które pozwalają na budowanie złożonych oraz interaktywnych aplikacji. Wykorzystanie języka
TypeScript do napisania kodu źródłowego aplikacji pozwoliło na używanie narzędzi
niedostępnych~w~standardowo używanym języku programowania aplikacji internetowych -- JavaScript. Do
stworzenia interfejsu użytkownika wykorzystana została biblioteka \textit{Chakra UI}. Udostępnia ona
gotowe komponenty, które po dodatkowym dostosowaniu pozwalają na pomięcie ręcznej stylizacji każdego
komponentu. Wykorzystano także dodatkowe, mniejsze biblioteki, które ułatwiły dodawanie wzorów
matematycznych oraz generowanie wykresów. Wszystkie wykorzystane narzędzia oraz biblioteki należą do
rodziny Otwartego Oprogramowania (ang. \textit{Open Source}) -- korzystanie~z~nich jest darmowe,
można edytować używany kod źródłowy według własnych potrzeb zgodnie~z~dołączona licencją.
