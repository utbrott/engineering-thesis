Celem pracy było stworzenie aplikacji internetowej, która może zostać
wykorzystana jako narzędzie do wspomagania studentów wykonujących ćwiczenia w laboratorium
Przetworników Pomiarowych.

Aplikacja została napisana w języku Typescript, który jest rozszerzeniem języka Javascript.
Zaimplementowane zostały trzy ćwiczenia, które wykonywane są przez studentów m.in. na zajęciach
,,Sensory i Aktuatory'' - do sensorów temperatury (RTD, Termoogniwa), przesunięcia (LVDT) oraz
tensometrów.

Aplikacja zbudowana została wykorzystując frameworki (ang. szkielety aplikacyjne) React oraz
Next.js, które pozwalają na budowanie złożonych oraz interaktywnych interfejsów
użytkownika, niż jest to możliwe wykorzystując tylko Javascript. Wykorzystanie języka Typescript
w~kodzie źródłowym aplikacji pozwoliło na wykorzystanie narzędzi niedostępnych w Javascript.

Interfejs użytkownika został zbudowany wykorzystując do tego biblioteki typu Open Source
(ang. Otwarte oprogramowanie) -- pozwoliło to na pominięcie wymogu ręcznej stylizacji każdego z
komponentów w aplikacji.