Celem pracy było zaprojektowanie oraz zaimplementowanie aplikacji internetowej wspomagającej
wykonywanie ćwiczeń~w~laboratorium Przetworników Pomiarowych.~W~ramach pracy zapoznano się~z~teorią
sensorów oraz przetworników pomiarowych. Opisany został proces projektowania~i~implementacji
aplikacji -- wykorzystane biblioteki oraz budowę kodu źródłowego.

Analiza teorii polegała na przyjrzeniu się stosowanym układom pomiarowym,~w~których używa się
wybranych do aplikacji sensorów. Zwrócono także uwagę na wzory potrzebne przy wykonywaniu obliczeń.
Dodatkowo przeanalizowane zostały typowe stosowane konfiguracje sensorów, tak aby
implementacja~w~narzędziu była jak najbliższa rzeczywistym przetwornikom pomiarowym.

Gotowa aplikacja oferuje możliwość konfigurowania sensorów pomiarowych. Zgodnie~z~założeniami
dostępne są zadania do wykonania~w~każdym~z~zaimplementowanych ćwiczeń laboratoryjnych. Rozwiązanie
posiada funkcję weryfikowania wpisywanych przez użytkownika odpowiedzi. Dodatkowo wszystkie
ćwiczenia posiadają zestaw generowanych charakterystyk przetwarzania, które użytkownik może
pobrać~i~zapisać na swoim komputerze~w~celu późniejszego wykorzystania.

Do implementacji wykorzystane zostały biblioteka \textit{React} oraz framework \textit {Next.js},
które pozwalają na tworzenie złożonych, interaktywnych aplikacji. Do stworzenia interfejsu
użytkownika wykorzystano bibliotekę \textit{Chakra UI} udostępniającą wiele gotowych
komponentów,~a~także dwie dodatkowe biblioteki -- \textit{React Latex} do wyświetlania wzorów
teoretycznych oraz \textit{Recharts} do generowania~i~wykreślania charakterystyk.

Aplikacja posiada interaktywny interfejs, który aktualizuje się~w~raz~z~działaniami prowadzonymi
przez użytkownika aplikacji. Całość interfejsu jest~w~j.~angielskim, co pozwala na wykorzystywanie
narzędzia nie tylko przez użytkowników~z~Polski. Wszystkie zaimplementowane funkcje -- generowanie
danych, weryfikacja poprawności wpisywanych odpowiedzi, generowanie wykresów działają prawidłowo.
Zostało to wykazane na podstawie przeprowadzonych testów.

Kolejnymi etapami rozwoju stworzonego rozwiązania są: poprawienie działania interfejsu użytkownika,
aby pełna funkcjonalność nie była ograniczona do wąskiego zakresu rozdzielczości ekranów
komputerów. Dodatkowo planowane jest przeniesienie elementów aplikacji, które zajmują się
wykonywaniem obliczeń do części ,,zaplecza'' (ang. \textit{backend}), aby były one niezależne od
przeglądarki internetowej,~z~której użytkownik korzysta do pracy~z~narzędziem.