Na testy poprawności działania interfejsu użytkownika składały się:
\begin{itemize}
  \item [--] sprawdzenie czy do momentu zapisania konfiguracji zablokowany jest widok zadań i
        możliwość wykonywania ich,
  \item [--] analiza czy podczas weryfikowania odpowiedzi aplikacja odrzuca próbę zatwierdzenia
        pustego pola lub niepoprawnej odpowiedzi,
  \item [--] weryfikacja czy do momentu ukończenia zadań zablokowane jest wyświetlanie wykresu.
\end{itemize}

\paragraph{Testy blokady widoczności zadań:} Na rysunku \ref{img:config-tasks}
przedstawionym~w~rozdziale \ref{sect:ui}, można zaobserwować, że przycisk ,,Apply'' jest aktywny, co
oznacza, że konfiguracja nie została zapisana. Widać także, że~w~komponentach do wyświetlania zadań
oraz przyjmowania odpowiedzi wyświetlają się ostrzeżenia~o~braku wybranej konfiguracji. Po
wciśnięciu przycisku zmienia się wygląd wyświetlanych komponentów, co przedstawiono na rysunku
\ref{img:tasks-unlocked}. Nie jest już możliwa zmiana konfiguracji, widoczna jest treść
zadań~i~możliwość wpisania odpowiedzi. Dodatkowo odblokowywuje się przycisk do resetowania
konfiguracji (wyświetlane jest nowe ostrzeżenie, tym razem po najechaniu kursorem na przycisk
,,Reset''). Na podstawie tego możliwe jest stwierdzenie, że funkcjonalność działa prawidłowo.

\addimage{0.7}{app-ui/tasks-unlocked}{\label{img:tasks-unlocked}Zmiany~w~interfejsie użytkownika po
  zapisaniu konfiguracji}

\paragraph{Testy poprawnego zachowania pól do wpisywania odpowiedzi:} Założeniem poprawnego
działania tego komponentu jest wyświetlanie odpowiednich wiadomości~o~błędzie (w przypadku próby
zatwierdzenia pustej odpowiedzi lub wpisania niepoprawnej wartości).

Rysunki \ref{img:answer-empty}, \ref{img:answer-wrong}, \ref{img:answer-correct} przedstawiają
odpowiednio reakcje aplikacji na brak wpisanej odpowiedzi, błędną oraz prawidłową
odpowiedź.~W~przypadku tego testu wartość, którą należało wpisać~w~pole to 8.13. Jak widać na
załączonych rysunkach wpisanie do pola 9.21 skutkowało wyświetleniem się informacji~o~błędnej
odpowiedzi. Natomiast w momencie wpisania i zatwierdzenia poprawnej odpowiedzi nie wyświetlany jest
żaden komunikat. Zmienia się jedynie wartość na liczniku powyżej pola. Pełni on w aplikacji funkcję
informowania o postępie w wykonywaniu ćwiczenia. Gdy wszystkie wymagane odpowiedzi zostaną
wprowadzone w miejscu licznika pojawi się komunikat oraz ikona sygnalizująca, że dane zadanie
zostało ukończone. Przedstawione zostało to na rysunku~\ref{img:all-answers}.

\begingroup
\addimage{0.5}{app-ui/answer-empty}{\label{img:answer-empty}Stan błędu oraz wiadomość~o~wymaganiu
  podania odpowiedzi}
\addimage{0.5}{app-ui/answer-wrong}{\label{img:answer-wrong}Stan błędu oraz komunikat~o~błędnej
  odpowiedzi}
\addimage{0.5}{app-ui/answer-correct}{\label{img:answer-correct}Zmiana licznika nad polem
  sygnalizująca zarejestrowanie poprawnej odpowiedzi}
\addimage{0.5}{app-ui/all-answers}{\label{img:all-answers}Pojawiający się w miejscu licznika
  komunikat o ukończeniu danego zadania}
\endgroup

\paragraph{Testy blokady widoczności wykresu:} Wykresy~w~aplikacji powinny być widoczne dopiero, gdy
użytkownik ukończy wszystkie przewidziane dla danego ćwiczenia laboratoryjnego zadania. Do tego
momentu ich widok jest zablokowany~i~wyświetla się ostrzeżenie, że zadania nie są ukończone. Na
rysunku \ref{img:tasks-not-done} przedstawiony został interfejs użytkownika po zapisaniu
konfiguracji -- można zauważyć, że widok wykresu jest zablokowany. Rysunek \ref{img:tasks-done}
przedstawia natomiast jak wygląda interfejs po ukończeniu zadań -- oba pola przyjmujące odpowiedzi
są zablokowane~i~posiadają komunikat~o~ukończeniu zadania. Na podstawie tego można stwierdzić, że
aplikacja prawidłowo blokuje widoczność wykresów do momentu ukończenia zadań.

\begingroup
\addimage{0.95}{app-ui/tasks-not-done}{\label{img:tasks-not-done}Interfejs
  użytkownika~z~zablokowanym widokiem wykresu, gdy zadania nie są ukończone}
\addimage{0.95}{app-ui/tasks-done}{\label{img:tasks-done}Interfejs użytkownika~z~widocznym
  wykresem po ukończeniu zadań przewidzianych dla danego ćwiczenia}
\endgroup