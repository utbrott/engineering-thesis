% Opis testów, które pozwalają określić czy aplikacja funkcjonuje jak powinna
Aplikacja posiada zaimplementowane trzy ćwiczenia laboratoryjne. Na każde~z~nich składają się:

\begin{itemize}
  \item [--] komponent do wyboru~i~zapisu konfiguracji sensora,
  \item [--] komponent do wyświetlania zadań oraz danych do nich,
  \item [--] komponent do przyjmowania odpowiedzi uzytkownika,
  \item [--] komponent do wyświetlania wygenerowanych wykresów.
\end{itemize}

Należało sprawdzić czy dodane funkcje obecne~w~kodzie źródłowym działają prawidłowo. Testy
podzielone zostały na dwie kategorie. Sprawdzone zostało czy zaimplementowane elementy interfejsu
użytkownika reagują poprawnie na zmiany stanu~w~aplikacji -- blokowanie widoku~w~odpowiednich
momentach interakcji użytkownika~z~aplikacją, wyświetlanie błędów jeżeli brakuje odpowiedzi lub jest
ona nieprawidłowa, etc.

Zweryfikowane zostało także czy obecne~w~aplikacji odzwierciedlenia teorii sensorów pomiarowych
zwracają poprawne wartości. Testy opierały się na sprawdzeniu czy wartości liczone przez aplikację
są zgodne~z~wartościami, które wyliczone zostały korzystając~z~wzorów teoretycznych. Dodatkowo
sprawdzone zostało czy generowane wykresy charakterystyk sensorów są prawidłowe.

\noindent Poniżej przedstawiono listę zweryfikowanych wykresów:

\begin{itemize}
  \item [--] \textbf{dla termorezystorów}: charakterystyki statyczna zależności rezystancji od
        temperatury oraz dynamiczna temperatury od czasu;
  \item [--] \textbf{dla termoogniw}: charakterystyki statyczna zależności napięcia wyjściowego od
        temperatury oraz dynamiczna temperatury od czasu;
  \item [--] \textbf{dla sensora LVDT}: charakterystyka zależności napięcia wyjściowego od
        przesunięcia rdzenia;
  \item [--] \textbf{dla tensometrów}: charakterystyki zależności napięcia wyjściowego od wartości
        odkształcenia, napięcia wyjściowego oraz rezystancji od temperatury;
\end{itemize}

\section{Poprawność działania interfejsu użytkownika}
Na testy poprawności działania interfejsu użytkownika składały się:
\begin{itemize}
  \item [--] sprawdzenie czy do momentu zapisania konfiguracji zablokowany jest widok zadań i
        możliwość wykonywania ich,
  \item [--] analiza czy podczas weryfikowania odpowiedzi aplikacja odrzuca próbę zatwierdzenia
        pustego pola lub nie poprawnej odpowiedzi,
  \item [--] weryfikacja czy do momentu ukończenia zadań zablokowane jest wyświetlanie wykresu.
\end{itemize}

\section{Poprawność wykonywanych przez aplikację obliczeń}
\section{Poprawność wykonywanych przez aplikację symulacji}