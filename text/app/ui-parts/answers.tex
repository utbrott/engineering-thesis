Komponent zbudowany jest wykorzystując \texttt{Form Control} z biblioteki Chakra UI. Jego zadaniem jest kontrola
formularza, który przyjmuje odpowiedzi użytkownika, jednakże wartości oraz weryfikacja ich nie są zarządzane lokalnie,
w komponencie. Zajmują się tym zewnętrzne funkcje przekazywane poprzez artybuty, znane jako \texttt{props} w
React \cite{react-docs}, które pozwalają komponentowi na odczytywanie obecnej wartości oraz zmianę ich. Lista atrybutów
przedstawiona została na rysunku \ref{lst:form-props}, natomiast wykorzystanie ich w komponencie przedstawiają
rysunki \ref{lst:form-part1}, \ref{lst:form-part2}, \ref{lst:form-part3}.

\addimage{0.9}{code/form-props}{\label{lst:form-props}Artybuty, \texttt{props}, komponentu}

\addimage{0.9}{code/form-part1}{\label{lst:form-part1}Wykorzystanie \texttt{props} w komponencie -- kontrola formularza,
 tytuł pola}

\addimage{0.9}{code/form-part2}{\label{lst:form-part2}Wykorzystanie \texttt{props} w komponencie -- zarządzanie polem
 formularza}

\addimage{0.9}{code/form-part3}{\label{lst:form-part3}Wykorzystanie \texttt{props} w komponencie -- przycisk do
 potwierdzania wpisywanej wartościq}