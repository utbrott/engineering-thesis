Komponent zbudowany jest wykorzystując \texttt{Form Control}~z~biblioteki \textit{Chakra UI}. Jego
zadaniem jest kontrola formularza, który przyjmuje odpowiedzi użytkownika, jednakże wartości oraz
ich weryfikacja nie są zarządzane lokalnie~w~komponencie. Zajmują się tym zewnętrzne funkcje
przekazywane poprzez artybuty, znane jako \texttt {props}~w~React \cite{react-docs}, które pozwalają
komponentowi na odczytywanie obecnej wartości oraz ich zmianę. Ponieważ do implementacji aplikacji
wykorzystany został język TypeScript każdemu artybutowi należało przypisać typ zmiennej lub funkcji,
które mogą być przez niego przekazane do komponentu. Lista atrybutów przedstawiona
została~w~listingu \ref{lst:form-props}, natomiast wykorzystanie ich~w~komponencie przedstawiają
listingi \ref{lst:form-part1}, \ref{lst:form-part2}, \ref{lst:form-part3}. Dodatkowo komponent
posiada takie samo ostrzeżenie jak ten wyświetlający zadania (co przedstawione zostało na
rysunku \ref{img:tasks-alert}). Pełna implementacja komponentu -- jakie funkcje przekazywane
są~z~zewnątrz -- przedstawiona została~w~rozdziale \ref{sect:laboratory}.

\addsnippet{code/form-props}{\label{lst:form-props}Artybuty, \texttt{props}, komponentu}

\addsnippet{code/form-part1}{\label{lst:form-part1}Wykorzystanie \texttt{props}~w~komponencie --
  kontrola formularza, tytuł pola}

\addsnippet{code/form-part2}{\label{lst:form-part2}Wykorzystanie \texttt{props}~w~komponencie --
  zarządzanie polem formularza}

\addsnippet{code/form-part3}{\label{lst:form-part3}Wykorzystanie \texttt{props}~w~komponencie --
  przycisk do potwierdzania wpisywanej wartości}

