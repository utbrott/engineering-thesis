Komponent zbudowany jest wykorzystując \texttt{Form Control}~z~biblioteki Chakra UI. Jego zadaniem
jest kontrola formularza, który przyjmuje odpowiedzi użytkownika, jednakże wartości oraz weryfikacja
ich nie są zarządzane lokalnie,~w~komponencie. Zajmują się tym zewnętrzne funkcje przekazywane
poprzez artybuty, znane jako \texttt {props}~w~React \cite{react-docs}, które pozwalają komponentowi
na odczytywanie obecnej wartości oraz zmianę ich. Ponieważ do implementacji aplikacji wykorzystany
został język TypeScript każdemu artybutowi należało przypisać typ zmiennej lub funkcji, które mogą
być przez niego przekazane do komponentu. Lista atrybutów przedstawiona została~w~listingu
\ref{lst:form-props}, natomiast wykorzystanie ich~w~komponencie przedstawiają listingi
\ref{lst:form-part1}, \ref{lst:form-part2}, \ref{lst:form-part3}. Pełna implementacja komponentu --
jakie funkcje przekazywane są~z~zewnątrz -- przedstawiona została~w~rozdziale \ref{sect:laboratory}.

\addsnippet{0.9}{code/form-props}{\label{lst:form-props}Artybuty, \texttt{props}, komponentu}

\addsnippet{0.9}{code/form-part1}{\label{lst:form-part1}Wykorzystanie \texttt{props}~w~komponencie --
  kontrola formularza, tytuł pola}

\addsnippet{0.9}{code/form-part2}{\label{lst:form-part2}Wykorzystanie \texttt{props}~w~komponencie --
  zarządzanie polem formularza}

\addsnippet{0.9}{code/form-part3}{\label{lst:form-part3}Wykorzystanie \texttt{props}~w~komponencie --
  przycisk do potwierdzania wpisywanej wartości}

