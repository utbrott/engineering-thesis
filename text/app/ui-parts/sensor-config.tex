Dla każdego sensora możliwa jest zmiana jego parametrów. Do implementacji wykorzystana została
metoda mapowania pól~w~całym komponencie na podstawie pliku konfiguracyjnego.
Odpowiadający za to kod przedstawiony został w listingu \ref{lst:config-field-map}. Dane pole
posiada dwa różne warianty do wyboru -- wybór~z~listy (wykorzystywany~w~przypadkach, gdy dany
parametr ma więcej opcji do wyboru) lub przycisk typu radio (dla przypadków, gdzie opcji do wyboru
jest nie więcej niż trzy). W listingu \ref{lst:config-field} przedstawiony został kod, który
renderuje pojedyncze pole na podstawie informacji zawartych~w~pliku konfiguracyjnym, którego
fragment przedstawiony został w listingu \ref{lst:config-field-data}.

\addsnippet{0.9}{code/config-field-map}{\label{lst:config-field-map}Funkcja generująca pola
  na podstawie pliku konfiguracyjnego}

\addsnippet{0.9}{code/config-field}{\label{lst:config-field}Kod komponentu pojedynczego pola
  konfiguracyjnego}

\addsnippet{0.9}{code/config-field-data}{\label{lst:config-field-data}Fragment pliku zawierającego
  dane do generowania pól konfiguracyjnych}

Plik konfiguracyjny ma formę tablicy zawierającej obiekty, które definują dane pole. Każdy obiekt
składa się~z~par klucz-wartość. Klucz \colortt{sensor} określa do jakiego laboratorium przynależą
definiowane wartości, \colortt{id} wykorzystywany jest przy manipulowaniu stanem, którego zadaniem
jest przechowywanie informacji~o~obecnej konfiguracji sensora, natomiast \colortt{type} pozwala na
zdefiniowanie jaki wariant pola ma zostać wyświetlony. Zarządzanie stanem oraz jak zostało to
rozwiązane~w~aplikacji opisane zostało~w~rozdziale \ref{sect:context}. Pozostałe klucze --
\colortt{label, options, optionLabels, defaultValue} wykorzystywane są do określenia wyświetlanego
tytułu pola, opcji dostępnych do wyboru oraz domyślnie wybranej opcji.