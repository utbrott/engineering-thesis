Aplikacja została zaprojektowana~w~taki sposób, że do momentu zapisania konfiguracji zadania nie są
wyświetlane oraz nie można przejść do ich wykonywania. Na rysunku \ref{img:tasks-alert}
przedstawione zostało rozwiązanie zastosowane~w~aplikacji -- ostrzeżenie informujące~o~tym, że
należy zapisać konfigurację, aby wyświetlić zadania. Listing \ref{lst:tasks-alert-code} przedstawia
natomiast implementację takiego ostrzeżenia~w~kodzie źródłowym aplikacji.

\addimage{0.65}{app-ui/tasks-alert}{\label{img:tasks-alert}Ostrzeżenie informujące~o~konieczności
  zapisania konfiguracji}

\addsnippet{0.9}{code/tasks-alert-code}{\label{lst:tasks-alert-code}Implementacja ostrzeżenia w
  kodzie.}

Do wyświetlania dodanych zadań wykorzystano metodę identyczną jak~w~przypadku komponentu
konfigurującego sensory tj. na podstawie mapowania tablicy,~w~tym przypadku przypisanej
do każdego~z~laboratoriów, zawierającej obiekty~z~polami klucz-wartość. Każdy obiekt posiada
informację o \colortt{taskId} (numerze), \colortt {content} (treści) zadania,~a~także flagę
\colortt{hasData}, która informuje kod renderujący komponent~o~tym, czy należy wyświetlić dodatkową
linię pod treścią, która zostanie uzupełniona wygenerowanymi danymi. Jedną~z~tablic przechowującą
informację do wyświetlania zadań przedstawia listing \ref{lst:task-prompts-config}, natomiast
implementację komponentu, którego zadaniem jest wyświetlanie zadań na podstawie tych informacji
przedstawia listing \ref{lst:task-prompts}.

\addsnippet{0.9}{code/task-prompts-config}{\label{lst:task-prompts-config}Tablica zawierające dane
  do wyświetlania zadań.}

\addsnippet{0.9}{code/task-prompts}{\label{lst:task-prompts}Implementacja komponentu dynamicznie
  wyświetlającego zadania do laboratorium.}