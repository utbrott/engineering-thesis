Generowanie wykresów~w~aplikacji wykorzystuje do tego bibliotekę \textit{Recharts}. Udostępnia ona
komponenty \texttt{LineChart, CartesianGrid, XAxis, YAxis, Line}, które połączone pozwalają na
wyświetlanie wykresów. Dane, które mają być wykreślone podawane są poprzez zewnętrzne atrybuty,
ponieważ generowane są~w~komponencie danego laboratorium (implementacja przedstawiona
została~w~rozdziale \ref{sect:laboratory}).

Dodatkowo wykorzystana została biblioteka \textit{recharts-to-png}, która pozwala na zapisywanie
wykresów jako obrazów~w~formacie PNG, aby można było je~w~późniejszym czasie wykorzystać,
przykładowo pisząc sprawozdanie~z~wykonanego ćwiczenia laboratoryjnego. Implementacja tej
funkcjonalności przedstawiona została~w~listingu \ref{lst:chart-download}.

\addsnippet{code/chart-download}{\label{lst:chart-download}Implementacja funkcjonalności
  zapisywania wykresów jako obrazy}

Aplikacja posiada dwa warianty komponetu do wyświetlania wykresu -- \texttt{SingleLineChart} oraz
\texttt{MultiLineChart}. Wykorzystywane są odpowiednio, tam gdzie potrzebny jest wykres, który
posiada tylko jedną linię lub który posiada wiele linii (w przypadku aplikacji jest to
laboratorium~z~sensorami temperatury, gdzie wykresy do charakterystyk dynamicznych posiadają trzy
linie).
