% Opisanie jak zbudowane jest jedno~z~laboratorium
Na kod źródłowy każdego ćwiczenia laboratoryjnego składają się wywołania opisanych~w~poprzednich
rozdziałach komponentów oraz specjalnych funkcji, znanych~w~React jako \texttt{hooks}. Budowanie
własnych funkcji typu \texttt{hook} pozwala na ekstrakcję logiki, która byłaby wielokrotnie
powtarzana~w~różnych miejscach aplikacji \cite{react-docs}. Przykładem może być odczytywanie
wartości~z~komponentu przyjmującego odpowiedzi użytkownika, której implementacja przedstawiona
została~w~listingu
\ref{lst:example-hook}.

\addsnippet{code/example-hook}{\label{lst:example-hook}Implementacja przykładowej funkcji typu
  \texttt{hook}}

Zaprezentowany własny hook wykorzystuje funkcję \texttt{useReducer} -- hook udostępniony przez
framework React. Pozwala on na zarządzanie złożonymi stanami, które składają się~z~podwartości lub
gdy jeden ze stanów zależny jest od innego obecnego~w~tej samej funkcji lub komponencie. Na
podstawie zmiennych wejściowych (w tym przypadku tablicy wartości do porównywania tego co jest
wpisywane~w~pole~z~wartością poprawną) zwracane są zaimplementowane flagi \colortt{isFieldEmpty,
  isFieldInvalid}. Ich zadaniem jest przechowywanie informacji~o~tym, czy pole jest puste oraz jeżeli
pole nie jest puste czy wpisana wartość jest zgodna~z~tą~z~porównania. Dodatkowo hook posiada dwie
funkcje \colortt{handleChange, handleReset}, których zadaniem jest odpowiednio aktualizowanie
wartości~w~pamięci aplikacji do najnowszej~w~polu oraz czyszczenie pola przy wywołaniu drugiej
funkcji.

Cały komponent zarządza wartościami dla danego ćwiczenia, które przechowywane są~w~Context oraz
przekazuje odpowiednie dane do komponentów, które takich informacji potrzebują
(przykładowo przekazanie informacji~o~wybranej konfiguracji~z~komponentu, który się tym zajmuje do
komponentu, który na ich podstawie generuje dane do zadań). Dodatkowo zajmuje się uzupełnianiem
informacji, które mają być wyświetlone użytkownikowi~w~oknie dialogowym~z~teorią do ćwiczenia
(na podstawie wczytanych plików konfiguracyjnych). Przedstawia to listing \ref{lst:modal-content}.

\addsnippet{code/modal-content}{\label{lst:modal-content}Fragment kodu zajmujący się
  populowaniem okna dialogowego teorii odpowiednimi wzorami~i~tabelami na podstawie plików
  konfiguracyjnych}

Komponent każdego ćwiczenia laboratoryjnego wykorzystuje dodane funkcje (hooki) do zarządzania
stanem, który wykorzystywany jest przez odpowiednie komponenty budujące interfejs użytkownika.
Poniżej przedstawione są zadania, które wykonują zaimplementowane hooki:

\begin{itemize}
  \item [--] \colortt{useFormInput} przyjmuje wartości wpisywane~w~pole odpowiedzi do zadania, w
        razie potrzeby ustawia odpowiednie flagi związane~z~porównaniem do poprawnej wartości,
  \item [--] \colortt{useFormValidation} na podstawie flag~z~powyższej funkcji aktualizuje liczbę
        poprawnie wpisanych odpowiedzi lub ustawia flagę~o~błędzie~i~wiadomość~z~nią związaną pod
        polem na odpowiedź,
  \item [--] \colortt{useCompleteTask} sprawdza czy liczba poprawnych odpowiedzi jest wystarczająca
        do uznania zadań jako ukończone~i~pozwala aplikacji na wyświetlenie wykresu,
  \item [--] \colortt{useSingleLineChartData} tworzy tablicę danych, która wykorzystywana jest do
        wykreślenia wykresu.
\end{itemize}
