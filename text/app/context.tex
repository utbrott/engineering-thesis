% Opisać jak działa przechowywanie danych~w~aplikacji, że nie istnieje tutaj baza danych, a
% wszystko jest przechowywane~w~stanach~i~zarządzane~z~wykorzystaniem React Context
Do zarządzania stanem komponentów wykorzystano React Context -- metodę pozwalającą na przekazywanie
danych~w~drzewie komponentów bez wykorzystywania artybutów (props) na każdym poziomie
\cite{react-docs}. Standardowa metoda, wykorzystująca zależność Rodzic-Dziecko (ang. \textit{Parent
  to child}) może być uciążliwa, ponieważ dane komponenty tworzące laboratorium składają się
dodatkowo~z~wielu mniejszych komponentów. Zastosowanie tej metody wymagałoby dodawania artybutów do
każdego~z~nich, tylko~w~celu przekazania danych~z~komponentu laboratorium do najniżej
położonego~w~drzewie elementu, gdzie dane te są wykorzystywane. Schematy blokowe przedstawiające
standardową metodę przekazywania danych oraz metodą korzystającą~z~Context przedstawiają odpowiednio
rysunki \ref{img:data-props} oraz \ref{img:data-context}.

\addimage{0.9}{code/data-props}{\label{img:data-props}Metoda tradycyjna przekazywania danych -
  atrybuty (props)}

\addimage{0.9}{code/data-context}{\label{img:data-context}Metoda wykorzystująca Context}

Jak widać na przedstawionych schematach, metoda wykorzystująca Context jest dużo bardziej optymalna.
Pozwala ona na pominięcie przekazywania danych, właściwości lub funkcji przez całe drzewo
komponentów. Zamiast tego wartość ta przekazywana jest bezpośrednio do komponentu, który~z~niej
korzysta, nawet jeżeli znajduje się~w~zupełnie innej części drzewa.

W~aplikacji Context używany jest~w~celu przekazywania:
\begin{itemize}
  \item[--] informacji~o~obecnie wybranej konfiguracji sensora dla danego laboratorium,
  \item[--] przekazywania funkcji do aktualizacji stanu zarządzającego konfiguracją sensora,
  \item[--] stanu czy konfiguracja jest zapisana (co pozwala na odblokowanie zadań),
  \item[--] wartości~z~pliku konfiguracyjnego~z~treściami zadań
  \item[--] obiektów zawierających tablice~z~danymi do generowania wartości
  \item[--] obiektów~z~tablicami danych do sprawdzania poprawności wpisywanej przez użytkownika
        odpowiedzi
\end{itemize}
Powyższe elementy przedstawione zostały~w~listingach \ref{lst:context-sensor},
\ref{lst:context-functions}.

\addsnippet{code/context-sensor}{\label{lst:context-sensor}Przykładowa fragment Contextu z
  informacjami dla konkretnego laboratorium}

\addsnippet{code/context-functions}{\label{lst:context-functions}Funkcje przekazywane globalnie
  przez Context~w~drzewie komponentów}