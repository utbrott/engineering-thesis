% Opisanie tylko wybranych funkcji jak one działają~i~pokazanie tego, że są po odzwiercidleniem
% teorii~w~formie kodu
Aplikacja korzysta~z~teorii sensorów pomiarowych (przedstawionej~w~rozdziale
\ref{sect:sensor-theory}) do generowania danych do zadań oraz weryfikowania poprawności odpowiedzi
wpisywanych przez użytkownika. Każde laboratorium posiada zestaw funkcji, które przekształcają wzory
teoretyczne do postaci kodu źródłowego.

Na funkcję składają się:
\begin{itemize}
  \item[--] zmienne wejściowe
  \item[--] wartości stałe dla danego wzoru teoretycznego,
  \item[--] wzór, który jest podzielony na mniejsze fragmenty,
  \item[--] połączenie części do postaci pełnego wzoru,
  \item[--] zaokrąglenie otrzymanej~z~funkcji wartości~i~zwrócenie jej.
\end{itemize}
Przykładowa funkcja, składająca się~z~powyższych elementów, przedstawiona została~w~listingu
\ref{lst:sensor-formula}.

\addsnippet{code/sensor-formula}{\label{lst:sensor-formula}Implementacja wzoru dla przetwornika
  LVDT~w~kodzie aplikacji}

Zmienne wejściowe, które przyjmuje dana funkcja pobierane są~z~Context, ponieważ tam przechowywane
są informacje~o~obecnie wybranej konfiguracji sensora. Wartości stałe dla każdego wzoru znajdują się
bezpośrednio~w~kodzie danej funkcji. Sam wzór, jeżeli składa się~z~wielu części, dzielony jest na
mniejsze dodatkowe funkcje lub stałe -- zapewnia to znacznie lepszą czytelność kodu oraz pewność, że
dana funkcja pełni tylko jedną rolę. Wyznaczona wartość jest~w~fazie końcowej jest zaokrąglana do
setnych części,~w~celu łatwiejszego operowania nią podczas weryfikacji odpowiedzi użytkownika.
Zaokrąglenie zostało wykorzystane, ponieważ znacznie łatwiej poprosić użytkownika~o~wyznaczenie
wartości~w~takiej postaci, niż~w~formacie pełnej wartości zmiennoprzecinkowej -- często do 15 miejsc
po przecinku.

Wykorzystanie własnych funkcji zaimplementowanych bezpośrednio~w~kodzie aplikacji daje pewność, że
wykorzystywane wzory są prawidłowym odzwierciedleniem teorii. Dodatkowo nie jest wymagane
korzystanie~z~zewnętrznych bibliotek oraz generowanie wartości poza aplikacją (implementacja lokalna
pracuje szybciej, nawet~w~przypadku złożonych obliczeń w stosunku do pobierania danych~z~serwerów,
co jest zależne od jakości połączenia~z~internetem jakim dysponuje użytkownik).