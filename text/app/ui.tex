% Opisanie tego jak zbudowane jest UI ogólnie, nawiązanie, że opisane zostaną tylko te najważniejsze
Całość kodu źródłowego aplikacji, która odpowiedzialna jest za renderowanie interfejsu użytkownika wykorzystuje
paradygmat programowania komponentowego -- pozwala na projektowanie, implementację oraz testowanie elementów aplikacji
niezależnie od siebie. Dodatkowo, jeżeli zaistnieje taka potrzeba, część kodu źródłowego może być stworzona~w~zupełnie
innym języku programowania, a~następnie dołączona do aplikacji, gdzie będzie idealnie współpracować~z~resztą obecnych
już komponentów \cite{component-programming}. Interfejs użytkownika~w~aplikacji jest~w~języku angielskim. Został
stworzony~w~taki sposób, aby finalna aplikacja mogła być wykorzystywana przez użytkowników nie tylko~z~Polski.
Interfejs użytkownika zbudowany został wykorzystując do tego bibliotekę \textit{Chakra UI}, która jest zbiorem
modularnych komponentów przydatnych do budowania złożonych aplikacji~w~React. Udostępnia ona takie komponenty
jak \texttt{HStack, VStack} pozwalające na ustawianie elementów na stronie, \texttt{Button, Radio, Select} -- gotowe
przyciski, kontrola formularzy oraz \texttt{Text, Heading} do manipulacji nagłówkami~i~tekstem na stronie.

% Opis poszególnych, ważnych składowych UI ze snippetami
\subsection{Elementy interfejsu}
Głównymi elementami interfejsu są komponenty odpowiedzialne za:
\begin{itemize}
  \item[--] konfigurację sensora dla danego laboratorium,
  \item[--] wyświetlanie zadań do wykonania oraz wygenerowanych danych,
  \item[--] przyjmowanie oraz weryfikowanie odpowiedzi wpisywanych przez użytkownika,
  \item[--] prezentowanie wzorów oraz danych niezbędnych do obliczeń,
  \item[--] wyświetlanie wygenerowanego wykresu.
\end{itemize}
Powyższe elementy przedstawione zostały na rysunkach \ref{img:config-tasks}, \ref{img:theory},
\ref{img:generated-graph}.

\begingroup
\addimage{0.65}{app-ui/config-tasks}{\label{img:config-tasks}Komponenty do konfiguracji,
  wyświetlania zadań~i~przyjmowania odpowiedzi}

\addimage{0.65}{app-ui/theory}{\label{img:theory}Komponent prezentujący wzory oraz dane}

\addimage{0.65}{app-ui/generated-graph}{\label{img:generated-graph}Komponent wyświetlający
  wygenerowany wykres}
\endgroup

\subsubsection{Konfiguracja sensorów}
Dla każdego sensora możliwa jest zmiana jego parametrów. Do implementacji wykorzystana została
metoda mapowania pól~w~całym komponencie na podstawie pliku konfiguracyjnego.
Odpowiadający za to kod przedstawiony został w listingu \ref{lst:config-field-map}. Dane pole
posiada dwa różne warianty do wyboru -- wybór~z~listy (wykorzystywany~w~przypadkach, gdy dany
parametr ma więcej opcji do wyboru) lub przycisk typu radio (dla przypadków, gdzie opcji do wyboru
jest nie więcej niż trzy). W listingu \ref{lst:config-field} przedstawiony został kod, który
renderuje pojedyncze pole na podstawie informacji zawartych~w~pliku konfiguracyjnym, którego
fragment przedstawiony został w listingu \ref{lst:config-field-data}.

\addsnippet{code/config-field-map}{\label{lst:config-field-map}Funkcja generująca pola
  na podstawie pliku konfiguracyjnego}

\addsnippet{code/config-field}{\label{lst:config-field}Kod komponentu pojedynczego pola
  konfiguracyjnego}

\addsnippet{code/config-field-data}{\label{lst:config-field-data}Fragment pliku zawierającego
  dane do generowania pól konfiguracyjnych}

Plik konfiguracyjny ma formę tablicy zawierającej obiekty, które definują dane pole. Każdy obiekt
składa się~z~par klucz-wartość. Klucz \colortt{sensor} określa do jakiego laboratorium przynależą
definiowane wartości, \colortt{id} wykorzystywany jest przy manipulowaniu stanem, którego zadaniem
jest przechowywanie informacji~o~obecnej konfiguracji sensora, natomiast \colortt{type} pozwala na
zdefiniowanie jaki wariant pola ma zostać wyświetlony. Zarządzanie stanem oraz jak zostało to
rozwiązane~w~aplikacji opisane zostało~w~rozdziale \ref{sect:context}. Pozostałe klucze --
\colortt{label, options, optionLabels, defaultValue} wykorzystywane są do określenia wyświetlanego
tytułu pola, opcji dostępnych do wyboru oraz domyślnie wybranej opcji.

\subsubsection{Wyświetlanie zadań oraz wygenerowanego zestawu danych}
Aplikacja została zaprojektowana~w~taki sposób, że do momentu zapisania konfiguracji zadania nie są
wyświetlane oraz nie można przejść do ich wykonywania. Na rysunku \ref{img:tasks-alert}
przedstawione zostało rozwiązanie zastosowane~w~aplikacji -- ostrzeżenie informujące~o~tym, że
należy zapisać konfigurację, aby wyświetlić zadania. Listing \ref{lst:tasks-alert-code} przedstawia
natomiast implementację takiego ostrzeżenia~w~kodzie źródłowym aplikacji.

\addimage{0.65}{app-ui/tasks-alert}{\label{img:tasks-alert}Ostrzeżenie informujące~o~konieczności
  zapisania konfiguracji}

\addsnippet{0.9}{code/tasks-alert-code}{\label{lst:tasks-alert-code}Implementacja ostrzeżenia w
  kodzie.}

Do wyświetlania dodanych zadań wykorzystano metodę identyczną jak~w~przypadku komponentu
konfigurującego sensory tj. na podstawie mapowania tablicy,~w~tym przypadku przypisanej
do każdego~z~laboratoriów, zawierającej obiekty~z~polami klucz-wartość. Każdy obiekt posiada
informację o \colortt{taskId} (numerze), \colortt {content} (treści) zadania,~a~także flagę
\colortt{hasData}, która informuje kod renderujący komponent~o~tym, czy należy wyświetlić dodatkową
linię pod treścią, która zostanie uzupełniona wygenerowanymi danymi. Jedną~z~tablic przechowującą
informację do wyświetlania zadań przedstawia listing \ref{lst:task-prompts-config}, natomiast
implementację komponentu, którego zadaniem jest wyświetlanie zadań na podstawie tych informacji
przedstawia listing \ref{lst:task-prompts}.

\addsnippet{0.9}{code/task-prompts-config}{\label{lst:task-prompts-config}Tablica zawierające dane
  do wyświetlania zadań.}

\addsnippet{0.9}{code/task-prompts}{\label{lst:task-prompts}Implementacja komponentu dynamicznie
  wyświetlającego zadania do laboratorium.}


\subsubsection{Przyjmowanie~i~weryfikacja odpowiedzi użytkownika}
Komponent zbudowany jest wykorzystując \texttt{Form Control}~z~biblioteki Chakra UI. Jego zadaniem
jest kontrola formularza, który przyjmuje odpowiedzi użytkownika, jednakże wartości oraz weryfikacja
ich nie są zarządzane lokalnie,~w~komponencie. Zajmują się tym zewnętrzne funkcje przekazywane
poprzez artybuty, znane jako \texttt {props}~w~React \cite{react-docs}, które pozwalają komponentowi
na odczytywanie obecnej wartości oraz zmianę ich. Ponieważ do implementacji aplikacji wykorzystany
został język TypeScript każdemu artybutowi należało przypisać typ zmiennej lub funkcji, które mogą
być przez niego przekazane do komponentu. Lista atrybutów przedstawiona została~w~listingu
\ref{lst:form-props}, natomiast wykorzystanie ich~w~komponencie przedstawiają listingi
\ref{lst:form-part1}, \ref{lst:form-part2}, \ref{lst:form-part3}. Pełna implementacja komponentu --
jakie funkcje przekazywane są~z~zewnątrz -- przedstawiona została~w~rozdziale \ref{sect:laboratory}.

\addsnippet{0.9}{code/form-props}{\label{lst:form-props}Artybuty, \texttt{props}, komponentu}

\addsnippet{0.9}{code/form-part1}{\label{lst:form-part1}Wykorzystanie \texttt{props}~w~komponencie --
  kontrola formularza, tytuł pola}

\addsnippet{0.9}{code/form-part2}{\label{lst:form-part2}Wykorzystanie \texttt{props}~w~komponencie --
  zarządzanie polem formularza}

\addsnippet{0.9}{code/form-part3}{\label{lst:form-part3}Wykorzystanie \texttt{props}~w~komponencie --
  przycisk do potwierdzania wpisywanej wartości}




\subsubsection{Prezentacja wzorów~i~danych potrzebnych do obliczeń}
Część teoretyczna do każdego ćwiczenia laboratoryjnego prezentowana jest~w~postaci okna dialogowego.
Taka implementacja została wykorzystana, aby użytkownik nie musiał zamykać strony~z~wykonywanym
ćwiczeniem (co skutkowałoby utratą postępów, ponieważ aplikacja nie posiada bazy danych). Zawartość
okna dialogowego konfigurowana jest przez dwa dodatkowe komponenty -- \texttt{Formula, Table},
których zadaniem jest generowanie odpowiednio listy wzorów oraz tabel~z~danymi na podstawie plików
konfiguracyjnych. Kod źródłowy odpowiadający za generowanie listy wzorów oraz tabel przedstawiony
jest~w~listingach \ref{lst:theory-formula} oraz \ref{lst:theory-table}, natomiast~w~listingach
\ref{lst:formula-config}, \ref{lst:table-config} przedstawione zostały fragmenty plików, które
zawierają informacje jakie wzory oraz tabele mają zostać wygenerowane przez komponenty.

\addsnippet{code/theory-formula}{\label{lst:theory-formula}Kod komponentu odpowiedzialnego za
  generowanie listy wzorów}

\addsnippet{code/theory-table}{\label{lst:theory-table}Komponent generujący tabele~z~danymi do
  ćwiczenia~w~oknie dialogowym}

\addsnippet{code/formula-config}{\label{lst:formula-config}Fragment pliku~z~informacjami o
  wzorach do wyświetlenia~w~oknie dialogowym}

\addsnippet{code/table-config}{\label{lst:table-config}Fragment pliku konfiguracyjnego do
  komponentu~z~tabelami}

Ponieważ pliki konfiguracyjne mają format tablicy~z~obiektami, to generowanie kolejnych wzorów oraz
wierszy tabeli wykonywane jest przez mapowanie kolejnych wartości. Odpowiednie miejsca~w~kodzie,
gdzie wstawione są nazwy zmiennych odczytujących klucze~z~konfiguracji, uzupełniane są pobranymi
informacjami.


\subsubsection{Wyświetlanie wygenerowanego wykresu}
Generowanie wykresów~w~aplikacji wykorzystuje do tego bibliotekę \textit{Recharts}. Udostępnia ona
komponenty \texttt{LineChart, CartesianGrid, XAxis, YAxis, Line}, które połączone pozwalają na
wyświetlanie wykresów. Dane, które mają być wykreślone podawane są poprzez zewnętrzne atrybuty,
ponieważ generowane one są~w~komponencie danego laboratorium (implementacja przedstawiona
została~w~rozdziale \ref{sect:laboratory}).

Dodatkowo wykorzystana została biblioteka \textit{recharts-to-png}, która pozwala na zapisywanie
wykresów jako obrazów~w~formacie PNG, aby można było je~w~późniejszym czasie wykorzystać,
przykładowo pisząc sprawozdanie~z~wykonanego ćwiczenia laboratoryjnego.

Aplikacja posiada dwa warianty komponetu do wyświetlania wykresu -- \texttt{SingleLineChart} oraz
\texttt{MultiLineChart}, wykorzystywane odpowiednio, tam gdzie potrzeba wykresu, który posiada tylko
jedną linię lub który posiada wiele linii (w przypadku aplikacji jest to laboratorium~z~sensorami
temperatury, gdzie wykresy do charakterystyk dynamicznych posiadają trzy linie)
